\section{Описание}
\subsection*{Параметризация}
Параметризация шара производится по двум углам --- $\phi$ и $\theta$ в сферической системе координат. Табулируя уравнение сферы по углам можно получить глобус, который при достаточной степени аппроксимации неотличим от шара.

\subsection*{Освещение}
Для построения фотореалистичного изображения буду использовать простую модель освещения:
\begin{equation*}
    I = I_a + I_d + I_s
\end{equation*}
где $I_a$ --- фоновая составляющая, $I_d$ --- рассеянная составляющая, $I_a$ --- зеркальная составляющая.

\subsection*{Фоновая составляющая}
Даже при отсутсвии света в комнате мы может различать объекты, потому что лучи отражаются от всех поверхностей в комнате. Расчёт всех отражений слишком сложный, поэтому для простоты каждый объект получает часть фонового освещения:
\begin{equation*}
    I_a = k_a \cdot i_a
\end{equation*}
где $k_a$ --- свойство материала воспринимать фоновое освещени, $i_a$ --- мощность фонового освещения.

\subsection*{Рассеянная составляющая}
Рассеянное отражение света происходит, когда свет как бы проникает под поверхность объекта, поглощается, а затем вновь испускается.
При этом положение наблюдателя не имеет значения, так как диффузно отраженный свет рассеивается равномерно по всем направлениям.

Интенсивность света обратно пропорциональна квадрату расстояния от источника, следовательно, объект, лежащий дальше от него, должен быть темнее. Для простоты буду применять модель $d + K$, так как в модели обратных квадратов для $|d| \leqslant 1$ будет осветление объекта, что нужно отдельно обрабатывать. Константа $K$ в этом выражении подбиратеся из соображений эстетики.

Выражение для рассеянного освещения имеет следующий вид:
\begin{equation*}
    I_d = {{k_d \cdot i_l} \over {d + K}} \cdot \cos(\vec{L}, \vec{N})
\end{equation*}
где $k_d$ --- свойство материала воспринимать рассеянное освещение, $i_l$ --- интенсивность точеченого источника, $\vec{L}$ --- направление из точки на источник света, $\vec{N}$ --- вектор нормали в точке, $K$ --- произвольная постоянная, $d$ --- расстояние от источника света до точки.

\subsection*{Зеркальная составляющая}
Интенсивность зеркально отраженного света зависит от угла падения, длины волны падающего света и свойств вещества отражающей поверхности. Зеркальное отражение света является направленным. Так как физические свойства зеркального отражения очень сложны, используется коээфициент глянцевости материала:
\begin{equation*}
    I_s = {{k_s \cdot i_l} \over {d + K}} \cdot {\cos ^ p}(\vec{R}, \vec{S})
\end{equation*}
где $k_s$ --- свойство материала воспринимать зекральное освещение, $\vec{R}$ --- вектор отражённого от поверхности луча, $\vec{S}$ --- вектор наблюдения, $p$ --- глянцевость материала.
\pagebreak
