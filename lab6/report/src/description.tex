\section{Описание}
Шейдер (Shader) --- программа для процессора графической карты (GPU), управляющая поведением шейдерной стадии графического конвейера и занимающаяся обработкой соответствующих входных данных.

В работе используются вершинный и фрагментный шейдеры. Первый трансформирует вершины из локального пространства в пространство камеры. Второй отвечает за закрасу геометрического объекта, интерполирует параметры и расчитывает цвет отдельно взятого пикселя.

Шейдеры можно писать на GLSL (Graphics Library Shader Language) --- высокоуровневом языке используемом для шейдеров OpenGL. Синтаксис языка очень похож на синтаксис C.

Для создания анимационного эффекта при каждой отрисовке шейдеру передаётся время, на основании которого программа пересчитывает цвет источника по синусоидальному закону.
\pagebreak
