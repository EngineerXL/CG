\section{Описание}
Кубическая B-сплайновая кривая определяется уравнением:

$r(t) = {{{1 - t} ^ 3} \over 6} \cdot P_0 + {{3 t ^ 3 - 6 t ^ 2 + 4} \over 6} \cdot P_1 + {{-3 t ^ 3 + 3 t ^ 2 + 3 t + 1} \over 6} \cdot P_2 + {{t ^ 3} \over 6} \cdot P_3$.

Матричная запись параметрического уравнения кривой:

$r(t) = P \cdot M \cdot T$,
$P =
\begin{pmatrix}
    x(t) \\
    y(t) \\
\end{pmatrix}$,
$M = {1 \over 6} \cdot
\begin{pmatrix}
    1 & -3 & 3 & -1 \\
    4 & 0 & 6 & 3 \\
    1 & 3 & 3 & -3 \\
    0 & 0 & 0 & 1 \\
\end{pmatrix}$,

$T =
\begin{pmatrix}
    t^0 \\
    t^1 \\
    t^2 \\
    t^3 \\
\end{pmatrix}$,
$P =
\begin{pmatrix}
    P_0 & P_1 & P_2 & P_3 \\
\end{pmatrix}$.

Матрица $M$ называется базисной матрицей B-сплайновой кривой.

Составная B-сплайновая кривая определяется объединением элементарных B-сплайновых кривых, составленных из $4$ точек. То есть для $6$ точек B-сплайн составляется из трёх элементарных B-сплайновых кривых. В таком случае уравнение кривой будет иметь вид:

$r_i(t) =
\begin{pmatrix}
    P_{i - 1} & P_{i} & P_{i + 1} & P_{i + 2} \\
\end{pmatrix}
\cdot M \cdot
\begin{pmatrix}
    (t - t_i) ^ 0 \\
    (t - t_i) ^ 1 \\
    (t - t_i) ^ 2 \\
    (t - t_i) ^ 3 \\
\end{pmatrix}$.
\pagebreak
