\section{Исходный код}

Файл \textit{Curve.cs} содержит код для генерации B-сплайна и методов изменения её параметров. В программе предусмотрено перемещение, добавление и удаление контрольных точек.

\begin{lstlisting}{language=[Sharp]C}
Matrix4D ConvertToMatrix(Vector4D v1, Vector4D v2, Vector4D v3, Vector4D v4)
{
    List<Vector4D> tmp = new List<Vector4D>(4);
    tmp.Add(v1);
    tmp.Add(v2);
    tmp.Add(v3);
    tmp.Add(v4);
    Matrix4D res = new Matrix4D();
    for (int i = 0; i < 4; ++i)
    {
        res[0, i] = tmp[i].X;
        res[1, i] = tmp[i].Y;
        res[2, i] = tmp[i].Z;
        res[3, i] = tmp[i].W;
    }
    return res;
}

public void CalcCurve()
{
    CountData = STEPS * (CountPoints - 3);
    Data = new List<Vector4D>(CountData);
    for (int i = 0; i < CountPoints - 3; ++i)
    {
        Vector4D p0 = Points[i];
        Vector4D p1 = Points[i + 1];
        Vector4D p2 = Points[i + 2];
        Vector4D p3 = Points[i + 3];
        Matrix4D tmp = ConvertToMatrix(p0, p1, p2, p3) * basisMatrix;

        for (int j = 0; j < STEPS; ++j)
        {
            double t = STEP * j;
            Vector4D tt = new Vector4D(1, t, Math.Pow(t, 2), Math.Pow(t, 3));
            Data.Add(tmp * tt);
        }
    }
}
\end{lstlisting}

\textit{Geometry.cs} содержит классы векторов и матрицы, методы для работы с ними.
В \textit{Misc.cs} заданы константы, используемые во всей программе.
Вся работа с интерфейсом, компиляция шейдеров, создание буфера вершин описаны в \textit{MainWindow.cs}.

\pagebreak
