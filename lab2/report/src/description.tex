\section{Описание}
Для параметризации правильной призмы нужно параметризовать правильный многоугольник, лежащий в основаниях призмы. Точки правильного многоугольника расположены на окружности, поэтому можно составить уравнение окружности и вычислять координаты вершин многоугольника.

Вся окружность составляет угол $2 \cdot \pi$ радиан, тогда для правильного $n$-угольника угол поворота между соседними вершинами $\Delta\phi = {{2 \cdot \pi} \over n}$, причём $\phi_0 = 0$.

После генерации многоугольника следует переместить его параллельно оси \textit{Oz} и соединить точки в правильном порядке, чтобы задать полигоны. Они должны быть заданы единообразно, чтобы вектора нормали смотрели в одну и ту же сторону относительно плоскости.
\pagebreak
