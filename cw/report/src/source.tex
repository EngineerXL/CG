\section{Исходный код}

Файл \textit{Curve.cs} содержит код для генерации NURBS-кривой 4 порядка. В ходе табулирования функциия, задающей кривую, вычисляются коэффициенты $N_{i,p}$ и координаты точки для какого-то $t$.

\begin{lstlisting}{language=[Sharp]C}
private double u(int i)
{
    return segments[i];
}

private double Nip(int i, int p, double t)
{
    if (p == 0)
    {
        return ((u(i) <= t) && (t < u(i + 1))) ? 1.0 : 0.0;
    }
    else
    {
        return (t - u(i)) / (u(i + p) - u(i)) * Nip(i, p - 1, t) + (u(i + p + 1) - t) / (u(i + p + 1) - u(i + 1)) * Nip(i + 1, p - 1, t);
    }
}

public void CalcCurve()
{
    GenSegments();
    CountData = steps * (CountPoints - DEGREE + 2);
    Data = new List<Vector4D>(CountData);
    double t = DEGREE - 1;
    for (int i = 0; i < CountPoints - DEGREE + 2; ++i)
    {
        for (int j = 0; j < steps; ++j)
        {
            Vector4D tt = new Vector4D();
            double qq = 0;
            for (int ii = 0; ii < CountPoints; ++ii)
            {
                double curNip = Nip(ii, DEGREE, t);
                tt = tt + curNip * W[ii] * Points[ii];
                qq = qq + curNip * W[ii];
            }
            Data.Add(tt / qq);
            t += step;
        }
    }
}
\end{lstlisting}
\pagebreak

В \textit{Figure.cs} задаётся сама фигура. Сначала вычисляется кривая с заданными параметрами, по этим точкам генерируется поверхность вращения. После генерации точек, образующих поверхность, задаются полигоны, вычисляются вектора нормали к вершинам.

\begin{lstlisting}{language=[Sharp]C}
public Figure(int paramPhi, int paramCurve, List<Vector4D> controlPoints, List<double> weights, Misc.Colour col)
{
    ParamPhi = paramPhi;
    ParamCurve = paramCurve - 1;
    DeltaPhi = Misc.ToRadians(Misc.MAX_DEG / (double)(ParamPhi));
    ControlPoints = controlPoints;
    Weights = weights;
    FigureColour = col;
    GenVertices();
    GenFigure();
    GenVertexNormals();
}

private void GenVertices()
{
    Vertices = new List<Vertex>();
    Curve nurbs = new Curve(ParamCurve, ControlPoints, Weights);
    LayerVertices = new List< List<Vertex> >();
    for (int i = 0; i < nurbs.CountData; ++i)
    {
        double curZ = nurbs.Data[i].Y;
        double curXY = (1 + nurbs.Data[i].X) / 2;
        double stepPhi = 0;
        List<Vertex> curLayerVertices = new List<Vertex>();
        for (int j = 0; j < ParamPhi; ++j)
        {
            Vertex curVertex = new Vertex(curXY * Math.Cos(stepPhi), curXY * Math.Sin(stepPhi), curZ, FigureColour);
            curLayerVertices.Add(curVertex);
            Vertices.Add(curVertex);
            stepPhi = stepPhi + DeltaPhi;
        }
        LayerVertices.Add(curLayerVertices);
    }
    CenterLow = new Vertex(0, 0, Vertices[Vertices.Count - 1].Data.Z, FigureColour);
    CenterHigh = new Vertex(0, 0, Vertices[0].Data.Z, FigureColour);
    Vertices.Add(CenterLow);
    Vertices.Add(CenterHigh);
    for (int i = 0; i < Vertices.Count; ++i)
    {
        Vertices[i].Id = i;
    }
}

private void GenFigure()
{
    Polygons = new List<Polygon>();
    GenCenter();
}

private void GenCenter()
{
    for (int i = 0; i < LayerVertices.Count - 1; ++i)
    {
        for (int j = 0; j < ParamPhi; ++j)
        {
            Vertex a = LayerVertices[i][j];
            Vertex b = LayerVertices[i][(j + 1) % ParamPhi];
            Vertex c = LayerVertices[i + 1][(j + 1) % ParamPhi];
            Vertex d = LayerVertices[i + 1][j];
            Polygons.Add(new Polygon(a, c, b));
            Polygons.Add(new Polygon(c, a, d));
        }
    }
}

public void GenVertexNormals()
{
    foreach (Vertex vert in Vertices)
    {
        vert.Normal = new Vector4D();
        vert.Normal.W = 0;
    }
    foreach (Polygon poly in Polygons)
    {
        Vector4D polyN = poly.NormalVector();
        polyN.Normalize();
        foreach (Vertex vert in poly.Data)
        {
            vert.Normal = vert.Normal + polyN;
            vert.Normal.W += 1;
        }
    }
    foreach (Vertex vert in Vertices)
    {
        vert.Normal.X = vert.Normal.X / vert.Normal.W;
        vert.Normal.Y = vert.Normal.Y / vert.Normal.W;
        vert.Normal.Z = vert.Normal.Z / vert.Normal.W;
        vert.Normal.W = 0;
        vert.Normal.Normalize();
    }
}
\end{lstlisting}
\pagebreak

\textit{Geometry.cs} содержит классы векторов и матрицы, методы для работы с ними.
В \textit{Misc.cs} заданы константы, используемые во всей программе.
Вся работа с интерфейсом, компиляция шейдеров, создание буфера вершин описаны в \textit{MainWindow.cs}.

Вершинный шейдер \textit{Phong.vert} преобразует локальные координаты в видовые, затем передаёт их фрагментному шейдеру.

\begin{lstlisting}{language=GLSL}
#version 150 core

in vec3 cord3f;
in vec3 col3f;
in vec3 norm3f;
out vec3 position;
out vec3 normal;
uniform mat4 proj4f;
uniform mat4 view4f;
uniform mat4 model4f;
uniform bool useSingleColor;
uniform vec3 singleColor3f;
uniform bool moveToCorner;

const vec4 cornerVec = vec4(0.85f, -0.85f, 0, 0);

void main(void) {
    vec4 vertexPos = vec4(cord3f, 1.0f);
    vertexPos = (proj4f * view4f * model4f) * vertexPos;
    vec4 vertexNormal = vec4(norm3f, 0.0f);
    vertexNormal = (view4f * model4f) * vertexNormal;
    position = vertexPos.xyz;
    normal = normalize(vertexNormal.xyz);
    gl_Position = vertexPos;
    if (moveToCorner) {
        gl_Position += cornerVec;
    }
}
\end{lstlisting}
\pagebreak

Фрагментный шейдер \textit{Phong.frag} реализует затенение Фонга, интерполирующее нормаль к точке.

\begin{lstlisting}{language=GLSL}
#version 150 core

in vec3 position;
in vec3 normal;
out vec4 color;
uniform bool useSingleColor;
uniform vec3 singleColor3f;
uniform vec3 ka3f;
uniform vec3 kd3f;
uniform vec3 ks3f;
uniform vec3 light3f;
uniform vec3 ia3f;
uniform vec3 il3f;
uniform float p;
uniform vec3 camera = vec3(0, 0, -1e9f);
uniform float k = 0.5;

void main(void) {
    if (useSingleColor) {
        color = vec4(singleColor3f, 1.0f);
        return;
    }
    vec3 n = normal;
    if (!gl_FrontFacing) {
        n = -n;
    }
    vec3 res = singleColor3f;
    vec3 l = light3f - position;
    float d = length(l);
    l = normalize(l);
    vec3 s = normalize(camera - position);
    vec3 r = normalize(2 * n * dot(n, l) - l);
    float diffusal = dot(l, n);
    float specular = dot(r, s);
    if (diffusal < 1e-3) {
        diffusal = 0;
        specular = 0;
    }
    if (specular < 1e-3) {
        specular = 0;
    }
    for (int i = 0; i < 3; ++i) {
        res[i] = res[i] * (ia3f[i] * ka3f[i] + il3f[i] * (kd3f[i] * diffusal + ks3f[i] * pow(specular, p)) / (d + k));
    }
    color = vec4(res, 1.0f);
}
\end{lstlisting}
\pagebreak
