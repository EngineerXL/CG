\section{Описание}
Для изображения кривой используется техника расчёта координат кривой при каких-то близких значениях $\phi$. Под шагом подразумевается модуль разности $|{\phi}_i - {\phi}_{i - 1}|$. Чем больше степень аппроксимации, тем меньше шаг.

Полученные координаты точек нужно перевести в экранные координаты, так как их начало находится в левом верхнем углу экрана и ось \textit{Oy} направлена вниз, а не вверх. С помощью библиотеки Cairo рисуем линии между соседними точками кривой, получая ломаную. При достаточной степени аппроксимации она визуально не отличима от кривой.
\pagebreak
