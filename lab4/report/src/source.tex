\section{Исходный код}

Файл \textit{Figure.cs} содержит код для генерации фигуры. После генерации точек, образующих поверхность, задаются полигоны, вычисляются вектора нормали к вершинам.

\begin{lstlisting}{language=[Sharp]C}
public Figure(int paramPhi, int paramTheta, double paramR, double maxTheta, Misc.Colour col)
{
    ParamPhi = paramPhi;
    ParamTheta = paramTheta - 1;
    R = paramR;
    MaxTheta = Misc.ToRadians(maxTheta);
    DeltaPhi = Misc.ToRadians(Misc.MAX_DEG / (double)(ParamPhi));
    DeltaTheta = MaxTheta / (double)(ParamTheta);
    FigureColour = col;
    GenVertices();
    GenFigure();
    GenVertexNormals();
}

private void GenVertices()
{
    CenterLow = new Vertex(0, 0, R * Math.Cos(MaxTheta), FigureColour);
    CenterHigh = new Vertex(0, 0, R, FigureColour);
    Vertices = new List<Vertex>();
    Vertices.Add(CenterLow);
    Vertices.Add(CenterHigh);
    LayerVertices = new List< List<Vertex> >();
    double stepTheta = DeltaTheta;
    for (int i = 0; i < ParamTheta; ++i)
    {
        double curZ = R * Math.Cos(stepTheta);
        double curXY = R * Math.Sin(stepTheta);
        double stepPhi = 0;
        List<Vertex> curLayerVertices = new List<Vertex>();
        for (int j = 0; j < ParamPhi; ++j)
        {
            Vertex curVertex = new Vertex(curXY * Math.Cos(stepPhi), curXY * Math.Sin(stepPhi), curZ, FigureColour);
            curLayerVertices.Add(curVertex);
            Vertices.Add(curVertex);
            stepPhi = stepPhi + DeltaPhi;
        }
        LayerVertices.Add(curLayerVertices);
        stepTheta = stepTheta + DeltaTheta;
    }
    for (int i = 0; i < Vertices.Count; ++i)
    {
        Vertices[i].Id = i;
    }
}

private void GenFigure()
{
    Polygons = new List<Polygon>();
    GenLow();
    GenHigh();
    GenSphere();
}

private void GenLow()
{
    int lastLayerInd = LayerVertices.Count - 1;
    for (int j = 0; j < ParamPhi; ++j)
    {
        Vertex a = LayerVertices[lastLayerInd][j];
        Vertex b = LayerVertices[lastLayerInd][(j + 1) % ParamPhi];
        Polygons.Add(new Polygon(CenterLow, b, a));
    }
}

private void GenSphere()
{
    for (int i = 0; i < ParamTheta - 1; ++i)
    {
        for (int j = 0; j < ParamPhi; ++j)
        {
            Vertex a = LayerVertices[i][j];
            Vertex b = LayerVertices[i][(j + 1) % ParamPhi];
            Vertex c = LayerVertices[i + 1][(j + 1) % ParamPhi];
            Vertex d = LayerVertices[i + 1][j];
            Polygons.Add(new Polygon(a, c, b));
            Polygons.Add(new Polygon(c, a, d));
        }
    }
}

private void GenHigh()
{
    for (int j = 0; j < ParamPhi; ++j)
    {
        Vertex a = LayerVertices[0][j];
        Vertex b = LayerVertices[0][(j + 1) % ParamPhi];
        Polygons.Add(new Polygon(CenterHigh, a, b));
    }
}

public void GenVertexNormals()
{
    foreach (Vertex vert in Vertices)
    {
        vert.Normal = new Vector4D();
        vert.Normal.W = 0;
    }
    foreach (Polygon poly in Polygons)
    {
        Vector4D polyN = poly.NormalVector();
        polyN.Normalize();
        foreach (Vertex vert in poly.Data)
        {
            vert.Normal = vert.Normal + polyN;
            vert.Normal.W += 1;
        }
    }
    foreach (Vertex vert in Vertices)
    {
        vert.Normal.X = vert.Normal.X / vert.Normal.W;
        vert.Normal.Y = vert.Normal.Y / vert.Normal.W;
        vert.Normal.Z = vert.Normal.Z / vert.Normal.W;
        vert.Normal.W = 0;
        vert.Normal.Normalize();
    }
}
\end{lstlisting}

\textit{Geometry.cs} содержит классы векторов и матрицы, методы для работы с ними.
В \textit{Misc.cs} заданы константы, используемые во всей программе.
\pagebreak

Вся работа с интерфейсом, компиляция шейдеров, создание буфера вершин описаны в \textit{MainWindow.cs}. Компиляция шейдеров:
\begin{lstlisting}{language=[Sharp]C}
private void CompileShaders()
{
    int[] success = new int[1];
    System.Text.StringBuilder txt = new System.Text.StringBuilder(512);

    vertexShader = gl.CreateShader(OpenGL.GL_VERTEX_SHADER);
    gl.ShaderSource(vertexShader, HelpUtils.ReadFromRes("Phong.vert"));
    gl.CompileShader(vertexShader);

    gl.GetShader(vertexShader, OpenGL.GL_COMPILE_STATUS, success);
    if (success[0] == 0)
    {
        gl.GetShaderInfoLog(vertexShader, 512, (IntPtr)0, txt);
        Console.WriteLine("Vertex shader compilation failed!\n" + txt);
    }

    fragmentShader = gl.CreateShader(OpenGL.GL_FRAGMENT_SHADER);
    gl.ShaderSource(fragmentShader, HelpUtils.ReadFromRes("Phong.frag"));
    gl.CompileShader(fragmentShader);

    gl.GetShader(fragmentShader, OpenGL.GL_COMPILE_STATUS, success);
    if (success[0] == 0)
    {
        gl.GetShaderInfoLog(fragmentShader, 512, (IntPtr)0, txt);
        Console.WriteLine("Fragment shader compilation failed!\n" + txt);
    }

    shaderProgram = gl.CreateProgram();
    gl.AttachShader(shaderProgram, vertexShader);
    gl.AttachShader(shaderProgram, fragmentShader);
    gl.LinkProgram(shaderProgram);

    gl.GetProgram(vertexShader, OpenGL.GL_LINK_STATUS, success);
    if (success[0] == 0)
    {
        gl.GetProgramInfoLog(vertexShader, 512, (IntPtr)0, txt);
        Console.WriteLine("Shader program linking failed!\n" + txt);
    }
}
\end{lstlisting}
\pagebreak

Вершинный шейдер \textit{Phong.vert} преобразует локальные координаты в видовые, затем передаёт их фрагментному шейдеру.

\begin{lstlisting}{language=GLSL}
#version 150 core

in vec3 cord3f;
in vec3 col3f;
in vec3 norm3f;
out vec3 position;
out vec3 normal;

uniform mat4 proj4f;
uniform mat4 view4f;
uniform mat4 model4f;
uniform bool useSingleColor;
uniform vec3 singleColor3f;

void main(void) {
    vec4 vertexPos = vec4(cord3f, 1.0f);
    vertexPos = (proj4f * view4f * model4f) * vertexPos;
    vec4 vertexNormal = vec4(norm3f, 0.0f);
    vertexNormal = (view4f * model4f) * vertexNormal;
    position = vertexPos.xyz;
    normal = normalize(vertexNormal.xyz);
    gl_Position = vertexPos;
}
\end{lstlisting}
\pagebreak

Фрагментный шейдер \textit{Phong.frag} реализует затенение Фонга, интерполирующее нормаль к точке.

\begin{lstlisting}{language=GLSL}
#version 150 core

in vec3 position;
in vec3 normal;
out vec4 color;

uniform bool useSingleColor;
uniform vec3 singleColor3f;
uniform vec3 ka3f;
uniform vec3 kd3f;
uniform vec3 ks3f;
uniform vec3 light3f;
uniform vec3 ia3f;
uniform vec3 il3f;
uniform float p;
uniform vec3 camera = vec3(0, 0, -1e9f);
uniform float k = 0.5;

void main(void) {
    if (useSingleColor) {
        color = vec4(singleColor3f, 1.0f);
        return;
    }
    vec3 res = singleColor3f;
    vec3 l = light3f - position;
    float d = length(l);
    l = normalize(l);
    vec3 s = normalize(camera - position);
    vec3 r = normalize(2 * normal * dot(normal, l) - l);
    float diffusal = dot(l, normal);
    float specular = dot(r, s);
    if (diffusal < 1e-3) {
        diffusal = 0;
        specular = 0;
    }
    if (specular < 1e-3) {
        specular = 0;
    }
    for (int i = 0; i < 3; ++i) {
        res[i] = res[i] * (ia3f[i] * ka3f[i] + il3f[i] * (kd3f[i] * diffusal + ks3f[i] * pow(specular, p)) / (d + k));
    }
    color = vec4(res, 1.0f);
}
\end{lstlisting}
\pagebreak
