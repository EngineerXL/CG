\section{Описание}
OpenGL (Open Graphics Library) --- спецификация, определяющая платформонезависимый программный интерфейс для написания приложений, использующих двумерную и трёхмерную компьютерную графику.

Для отображения на экран необходимо использовать Vertex Buffer. Этот метод хранит инфморацию об объекте непосредственно в видеопамяти. Напрямую обратиться к ней нельзя, OpenGL предосталвяет возможность создания буферов вершин и индексов, куда следует скопировать данные об объекте.

Шейдер (Shader) --- программа для процессора графической карты (GPU), управляющая поведением шейдерной стадии графического конвейера и занимающаяся обработкой соответствующих входных данных.

В работе используются вершинный и фрагментный шейдеры. Первый трансформирует вершины из локального пространства в пространство камеры. Второй отвечает за закрасу геометрического объекта, интерполирует параметры и расчитывает цвет отдельно взятого пикселя.

Шейдеры можно писать на GLSL (Graphics Library Shader Language) --- высокоуровневом языке используемом для шейдеров OpenGL. Синтаксис языка очень похож на синтаксис C.
\pagebreak
