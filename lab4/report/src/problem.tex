\CWHeader{Ознакомление с технологией OpenGL}

\CWProblem{
Создать графическое приложение с использованием OpenGL. Используя результаты предыдущей лабораторной работы, изобразить заданное тело с использованием средств OpenGL 2.1. Использовать буфер вершин. Точность аппроксимации тела задается пользователем. Обеспечить возможность вращения и масштабирования многогранника и удаление невидимых линий и поверхностей. Реализовать простую модель освещения на GLSL.

Параметры освещения и отражающие свойства материала задаются пользователем в диалоговом режиме.

\textbf{Вариант задания:} Шаровой сектор.
}
\pagebreak
